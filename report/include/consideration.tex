\section{考察}\label{sec:consideration}
鍵のオンライン化システムの有用性として,遠隔地からの錠の施錠確認,
多人数の共有スペースにおいて多数の鍵の擬似的生成など
が考えられ,今回の制作物も十分にその有用性を持っている.

課題としては,モバイルバッテリーの連続稼働時間が24時間と短いことから,
実際に利用する際はコンセントからの電源供給が必要になってしまうという
ものがある.これはモータを駆動するマイコンに電池消費の大きい
Raspberry Piを採用したことが一つの原因であると考えられ,
Webアプリケーションのためのサーバーを別で用意し,機能は制限されても
消費電力のより小さいマイコンを利用することが改善として必要だと言える.

