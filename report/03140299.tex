%        File: 03140299.tex
%     Created: Sun Feb 15 04:00 AM 2015 J
% Last Change: Sun Feb 15 04:00 AM 2015 J
%
\documentclass[a4paper,10pt]{jarticle}

%%%%%%%%%%%%%%%%%%%%%
% to input Japanese %
%%%%%%%%%%%%%%%%%%%%%
\usepackage[japanese]{babel}

%%%%%%%%%%%
% unknown %
%%%%%%%%%%%
% \usepackage{ascmac}

%%%%%%%%%%%%%%%%%%%%%%%%%%
% to be standard a4paper %
%%%%%%%%%%%%%%%%%%%%%%%%%%
\usepackage{geometry}
\geometry{left=20mm, right=20mm, top=20mm, bottom=40mm}

%%%%%%%%%%%%%%%%%%%%%
% to insert figures %
%%%%%%%%%%%%%%%%%%%%%
\usepackage[dvipdfmx]{graphicx}

%%%%%%%%%%%%%%%%%%%%%%%%%%
% to insert source codes %
%%%%%%%%%%%%%%%%%%%%%%%%%%
% \usepackage{listings, jlisting}
% \renewcommand{\lstlistingname}{list}
% \lstset{language=C,
%   basicstyle=\ttfamily\scriptsize,
%   commentstyle=\textit,
%   classoffset=1,
%   keywordstyle=\bfseries,
%   frame=tRBl,
%   framesep=5pt,
%   showstringspaces=false,
%   numbers=left,
%   stepnumber=1,
%   numberstyle=\tiny,
%   tabsize=2
% }

%%%%%%%%%%%%%%%%%%
% title & author %
%%%%%%%%%%%%%%%%%%
\title{自主プロジェクトレポート \\ keyopener - 鍵の遠隔操作システム}
\author{03-140299 東京大学機械情報工学科 3年 和田健太郎}


% 様式
% ----
% ・各自レポートを提出(A4用紙、左綴じ).
% ・テーマ名 ・氏名 ・番号を表紙に記入.
% ・目的(何をするものか)、原理、設計、製作過程、動作結果.
% ・工夫、苦労、自己評価、などの考察および感想を要領よくまとめてください.
% ・図表を充分に活用して、簡潔・具体的に伝わるように

\begin{document}
\maketitle

\section{概要}
近年スマートフォンなどのネットワークに常時接続される
デバイスの増加に伴い, あらゆるモノがオンライン化する
「モノのインターネット」 (IoT = Internet of Things)
に対する注目が高まっている.~\cite{intro-iot}

自主プロジェクトの約一ヶ月の製作期間で私はIoTデバイスをテーマとして,
玄関鍵をオンライン化し,遠隔操作できるシステムの開発に取り組んだ.
主な制作物は錠を開閉するデバイスと
それを遠隔操作する際のアプリケーションである.



\section{目的}
本プロジェクトでは鍵のオンライン化システムの開発を行い,
その有用性と課題を検討する.

製作物の機能は,以下のようである.

\begin{itemize}
  \item 取り付けが容易である
  \item 遠隔操作で鍵の開閉が可能である.
  \item 鍵が共有可能である.
\end{itemize}



\section{原理}
システムは大きく二つの部分から成り,
扉に取り付け鍵を回転するデバイスと,
それを遠隔する際の通信部分である.

鍵を回転するデバイスはドアの内側に取り付け,モータによって鍵の開閉を行う.
遠隔操作を行うために,モータを操作するマイコンはWifiを経由してネットワーク
に接続し,外部の電子機器から操作する.



\section{設計}
% システム図

\section{製作過程}
% 設計
% 工作
% ソフトウェア

\section{動作結果}
% 回転した

\section{考察}
\section{最後に}
% 感想など


\begin{thebibliography}{9}
  \bibitem{intro-iot} Cisco inSPire, 2013, http://cisco-inspire.jp/issues/0010/cover\_story.html
\end{thebibliography}


\end{document}
